% Appendix A
% For referencing this appendix elsewhere, use \ref{AppendixA}
\chapter{Methods} % Main appendix title
\label{Appendix-methods} 

This appendix details procedures used to carry out the experiments

\section{conda}
\label{app:methods:Conda}

conda commands used in this project:
% Another good simplified reference
% https://docs.conda.io/projects/conda/en/latest/commands.html

\begin{verbatim}
# List conda environments
$ conda info --envs
# Activate conda - conda environment appears in parenthesis on command prompt
$ conda activate
# Deactivate conda
$ conda deactivate
# Create and activate environment
$ conda create --name cleargrasp
$ conda activate cleargrasp
# list modules
$ conda list
\end{verbatim}

\section{ClearGrasp Install - Ubuntu 19.04}

For this project, ClearGrasp was originally deployed to an AWS (\cite{amazon2015amazon}) cloud EC2 server at the proof-of-concept stage, following the procedure described in \cite{cleargrasp-install2020}. For the actual work, using the Zivid One+ camera, ClearGrasp was deployed to a physical workstation: Dell Precison Tower 5810 with a 12-core Intel Xeon processor E5-1600 v3 and 32MB RAM. Given some software was already present, the procedure was simplified to:
\begin{verbatim}
# create conda environment
$ conda create --name cleargrasp
$ conda activate cleargrasp
# change python simlynk to point to v3.6.9
$ sudo ln -s /usr/bin/python3.6 /usr/bin/python
# install libraries
$ sudo apt-get install -y libhdf5-10 libhdf5-serial-dev libhdf5-dev libhdf5-cpp-11
$ sudo apt install -y libopenexr-dev zlib1g-dev openexr
$ sudo apt install -y xorg-dev  # display widows
$ sudo apt install -y libglfw3-dev
# clone repository
$ git clone git@github.com:Shreeyak/cleargrasp.git
$ cd cleargrasp
# download model checkpoints
$ wget http://clkgum.com/shreeyak/cleargrasp-checkpoints 
$ mv cleargrasp-checkpoints cleargrasp-checkpoints.zip
$ unzip cleargrasp-checkpoints.zip
$ mv cleargrasp-checkpoints data
# requirements
$ wget https://raw.githubusercontent.com/dsikar/cleargrasp/master/requirements_v2.txt
$ pip install -r requirements_v2.txt
# depth2depth compilation as per step 4 in https://github.com/Shreeyak/cleargrasp
# Install open3d
$ pip install open3d --no-cache-dir
# copy config file
$ cd eval_depth_completion/
$ cp config/config.yaml.sample config/config.yaml
# run evaluation
$ python eval_depth_completion.py -c config/config.yaml
\end{verbatim}

\section{ClearGrasp Install - Ubuntu 20.04}

Check Python version:
\begin{verbatim}
$ python --version
Python 3.8.5    
\end{verbatim}
Since Python is version 3.x, the symlink will not be required. Download Anaconda:
\begin{verbatim}
$ wget https://repo.anaconda.com/archive/Anaconda3-2020.11-Linux-x86_64.sh    
\end{verbatim}
Install:
\begin{verbatim}
$ bash Anaconda3-2020.11-Linux-x86_64.sh    
\end{verbatim}
Terminal restart (required), then, as with Ubuntu 19.04 install:
\begin{verbatim}
# List conda environments
$ conda info --envs
# Create and activate environment
$ conda create --name cleargrasp
$ conda activate cleargrasp
\end{verbatim}
And continue with setup:
\begin{verbatim}
# update and install
$ sudo apt update
$ sudo apt --fixed-broken install
\end{verbatim}
There are a couple of libraries missing:
\begin{verbatim}
E: Unable to locate package lbhdf5-10
E: Unable to locate package libhdf5-cpp-11
\end{verbatim}
This could be a problem. Next we run a modified version of the requirements\_v2.txt (dsikar/cleargrasp.git). Also threw an error, fixed by installing opencv-python and commenting out corresponding line in requirements\_v2.txt:
\begin{verbatim}
$ pip install opencv-python
opencv-python==4.1.1.26

\end{verbatim}

\section{Unity Robotics packages}

This section covers the installation of the Unity Robotics packages. As described in \url{https://github.com/Unity-Technologies/Unity-Robotics-Hub}, using version 2020.2.0+. This is done in Ubuntu by obtaining the Unity Hub link at \url{https://unity3d.com/get-unity/download/archive}, and starting unity from the command line, with the unityhub url as a parameter:
\begin{verbatim}
$ sudo ./UnityHub.AppImage unityhub://2020.2.3f1/8ff31bc5bf5b --no-sandbox 
\end{verbatim}







