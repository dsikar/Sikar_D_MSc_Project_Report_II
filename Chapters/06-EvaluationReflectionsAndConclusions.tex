%%%%%%%%%%%%%%%%%%%%%%%%%%%%%%%%%%%%%%%%%%%%
% EVALUATION, REFLECTIONS AND CONCLUSIONS %%
%%%%%%%%%%%%%%%%%%%%%%%%%%%%%%%%%%%%%%%%%%%%

%This chapter should evaluate the project work as a whole. Here the original choice of objectives, the literature examined, the methods used, the planning, etc. are all reviewed to see what has been achieved by undertaking the project. There may be a summary of general conclusions drawn from the work done, highlighting the particular contribution of your project. You should also consider the implications of these conclusions. Discuss any proposals that you might make for further work, having discovered what you now know. It is also important to include a reflective section covering what you have learned from the project process. What would you do differently if you were to start again, knowing what you now know? Your report MUST include adequate Evaluation, Reflections and Conclusions to gain a passing grade. 

\chapter{Evaluation, Reflections and Conclusions}
\label{Eval}

\begin{itemize}
    \item On the difficulty of obtaining datasets
    \item On the importance of being able to generate datasets, through packages such as Blender or methods such as Generative Adversarial Networks
    \item on the chance of direction from initial point cloud proposal to clear object detection
    \item on characteristics of Zivid One+ and RealSense 415 such as Stereo Camera Baseline (distance between stereo camera centres) and implications on this and future studies
\end{itemize}

\lipsum[1]
